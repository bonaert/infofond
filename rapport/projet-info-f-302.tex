\documentclass[a4paper]{article}

%% Language and font encodings
\usepackage[english]{babel}
\usepackage[utf8x]{inputenc}
\usepackage[T1]{fontenc}

%% Sets page size and margins
\usepackage[a4paper,top=3cm,bottom=2cm,left=3cm,right=3cm,marginparwidth=1.75cm]{geometry}

%% Useful packages
\usepackage{amsmath}
\usepackage{graphicx}
\usepackage[colorinlistoftodos]{todonotes}
\usepackage[colorlinks=true, allcolors=blue]{hyperref}

\title{Projet INFO-F-302 : Gestion des horaires de train}
\author{MAHIA Jérôme, BONAERT Gregory}

\begin{document}
\maketitle

\section{Sémantique des variables booléennes}

\begin{itemize}
\item $sur\_voie$ 
\\
\\
La variable booléenne $sur\_voie$
\item $dans\_gare$
\\
\\
La variable booléenne $dans\_gare$
\item $fait\_trajet$
\\
\\
La variable booléenne $fait\_trajet$
\end{itemize}


\section{Formules des contraintes}

\subsection{Contraines explicites}
\begin{itemize}


\item{Pour toute paire de gare (g, g'), pour chaque fenêtre horaire de durée TimeWindow,
il existe un train qui dessert g dans cette fenêtre, puis, plus tard, ce même train
desservira g', après une durée de trajet d’au plus TravelDuration.}

    \begin{equation*}
    \begin{split}
      & 
      \bigwedge_{(g,g'), g \neq g'} 
      \bigwedge_{TimeWindow [i_s,i_e]}
      \bigvee_{i \in [i_s,i_e]}
      \bigvee_{i < i' \leq min(TimeSlots - 1, i + TravelDuration)}
      \bigvee_{t}
      \big[T_{t,i,g,i', g'}] \\
    \end{split}
    \end{equation*}

Les Time Windows utilisées sont non-glissantes (elles ne se superposent pas).
    
    

\item{Pour prévenir tout risque de collision, une voie de chemin de fer entre deux gares,
appelé segment, peut accueillir au plus un train.}

    \begin{equation*}
    \begin{split}
      & \bigwedge_{t_1 \neq t_2}
      \bigwedge_{i}
      \bigwedge_{(g,g'), g \neq g'} \big[
      \neg V_{t_1,i,(g,g')} \lor \neg V_{t_2,i,(g,g')} \big] \\
    \end{split}
    \end{equation*}

\item{Les trains de type slow sont des omnibus : ils sont tenus de s’arrêter dans toutes les
gares. Les trains de type fast ne sont tenus de s’arrêter que dans les gares de type big.}

    \begin{equation*}
    \begin{split}
      & \bigwedge_{t}
      \bigwedge_{i}
      \bigwedge_{(g_1,g_2), (g_2, g_3)}  \big[
      \neg V_{t,i,(g_1,g_2)} \lor \neg V_{t,i+1,(g_2,g_3)} \lor (Rapide(t) \land Petite(g_2) \big] \\
    \end{split}
    \end{equation*}

\item{Afin de permettre aux usagers de monter à bord, tout train desservant une gare doit
le faire durant une durée d’au moins TimeWait.}

    \begin{equation*}
    \begin{split}
      & \bigwedge_{t, g, i} \bigwedge_{i+1 \leq i_2<i+TWait} \big[
      \big(G_{g, t, i-1} \lor \neg G_{g, t, i} \lor G_{g, t, i_2} \big) \\
    \end{split}
    \end{equation*}

\item{Les trains doivent respecter les temps de trajet prévus par l’infrastructure entre deux
gares. On supposera constant, les temps de trajet entre deux gares, ils seront donc donnés en entrée du problème.}

    \begin{equation*}
    \begin{split}
      & \bigwedge_{(g_1, g_2), t, i} \bigwedge_{i+1<=i_2<i+TravelDuration} \big[ V((g_1, g_2), t, i-1) \lor \neg V((g_1, g_2), t, i) \lor V((g_1, g_2), t, i_2)\big] \\
    \end{split}
    \end{equation*}

\item{Une gare ne peut accueillir plus de trains que ce que sa capacité le permet.}

    \begin{equation*}
    \begin{split}
        & \bigwedge_{g, i} \bigwedge_{Ensemble De Trains E \{t_1, t_2, \cdot, t_{cg}, t_{c_g+1}\}} \bigvee_{t \in E} \neg G_{t, g, i}
    \end{split}
    \end{equation*}
\end{itemize}

\subsection{Contraintes implicites}

\begin{itemize}

\item Tout train doit être quelque part

\begin{equation*}
    \begin{split}
      & \bigwedge_{t, i}
      \big[\bigvee_{g} \big(G_{g, i, t} \big) \lor \bigvee_{(g_1, g_2)} \big(V_{(g_1, g_2) i, t} \big) \big] \\
    \end{split}
    \end{equation*}


\item Un train ne peut pas être dans 2 gares différentes au même moment

\begin{equation*}
    \begin{split}
      & \bigwedge_{g_1, g_2, i>0, t, g_1 \neq g_2 }
      \big[\neg G_{g_1, i, t} \lor \neg G_{g_2, i, t} \big] \\
    \end{split}
    \end{equation*}
    
\item Un train ne peut pas être sur 2 segment différents au même moment

\begin{equation*}
    \begin{split}
      & \bigwedge_{(g_1, g_2), (g_3, g_4), i>0, t, g_1 \neq g_3, g_2 \neq g_4 }
      \big[\neg V_{(g_1, g_2), i, t} \lor \neg V_{(g_3, g_4), i, t} \big] \\
    \end{split}
    \end{equation*}

\item Un train ne peut pas être sur un segment et une gare au même moment

\begin{equation*}
    \begin{split}
      & \bigwedge_{(g_1, (g_2, g_3), i>0, t}
      \big[\neg G_{g_1, i, t} \lor \neg V_{(g_2, g_3), i, t} \big] \\
    \end{split}
    \end{equation*}


\item Lorsqu'on sort d'un segment A-B, soit on arrive à gare B soit on est sur un segment B-C

\begin{equation*}
    \begin{split}
      & \bigwedge_{(g_1, g_2), t, i}
      \big[
      \big(\neg V_{(g_1, g_2), i-1, t} \lor V_{(g_1, g_2), i, t} \big)
      \lor
      \big(G_{g_2, t, i} \lor \bigvee_{g_3,\ g_3\ \text{liée à}\ g_2} \big(V_{(g_2, g_3), i, t}\big) \big)
      \big]
      \end{split}
    \end{equation*}


\item Lorsqu'un train sort d'une gare A, il doit être sur un segment qui part de A

\begin{equation*}
    \begin{split}
      & \bigwedge_{g_1, t, i}
      \big[
      \big(\neg G_{g_1, i-1, t} \lor G_{(g_1, i, t} \big)
      \lor
      \big(\bigvee_{g_2,\ g_1\ \text{liée à}\ g_2} V_{(g_1, g_2), i, t}\big) 
      \big]\\
    \end{split}
    \end{equation*}
    
\item Contraintes supplémentaires afin de rendre la variable booléenne $fait\_trajet$ cohérente

Première partie \\
\begin{equation*}
    \begin{split}
      & \bigwedge_{t, g, g', i, i', g \neq g', i < i'}
      (\neg T_{t,i,g,i', g'} \lor G_{g, i, t} ) \land (\neg T_{t,i,g,i', g'} \lor  G_{g', i', t} )\\
    \end{split}
    \end{equation*}

Deuxième partie \\

\begin{equation*}
    \begin{split}
      & \bigwedge_{t, g, g', i, i', g \neq g', i < i'}
      \neg G_{g, i, t} \lor \neg G_{(g', i', t} \lor T_{t,i,g,i', g'} \\
    \end{split}
    \end{equation*}


\end{itemize}

%\section{?Contraintes retirées?}

\section{Questions supplémentaires}

\subsection{Remplacement de la première contrainte}

Remplaçons la première contrainte par: \\

Pour toute paire de gare $(g, g')$, pour chaque fenêtre horaire de durée $TimeWindow$,
il est possible de prendre un train dans la fenêtre horaire, en gare $g$, pour aller vers $g'$
après une durée maximale de trajet $TravelDuration$, le tout en changeant potentiellement de train au plus $NbChange$ fois. La version précédente du problème correspond
au cas $NbChange=0$.

\subsection{Proposer deux améliorations rendant le modèle plus réaliste.}

\subsubsection{Faire que les time windows puissent se superposer}

Si on permet aux time windows de se superposer (fenêtres glissantes), cela permet d'assurer un traffic plus fluide et efficace.

Pour modéliser cette amélioration, il faut modifier la contrainte 1 de façon à ce que les time windows utilisées puissent se superposer. Voici la modélisation:

   \begin{equation*}
    \begin{split}
      & 
      \bigwedge_{(g,g'), g \neq g'} 
      \bigwedge_{TimeWindow [i_s,i_e]}
      \bigvee_{i \in [i_s,i_e]}
      \bigvee_{i < i' \leq min(TimeSlots - 1, i + TravelDuration)}
      \bigvee_{t}
      \big[T_{t,i,g,i', g'}] \\
    \end{split}
    \end{equation*}
    
Les TimeWindows acceptées sont toutes les timeWindows tel que 
$$0 <= i_s, i_e = i + TimeWindow < TimeSlots$$

\subsubsection{Limiter la quantité de retard admise ou autoriser le retard}

Par défaut on autorise un retard illimité sur les voies, ce qui n'est pas très réaliste. Pour améliorer cela, on autorise au maximum 100\% de retard. Par exemple, si il faut 20 minutes pour parcourrir une voie, on autorise au maximum un temps de trajet de 40 minutes.

Autrement dit, si la voie a comme temps de trajet minimum TravelDuration, le train peut rester au maximum 2 x TravelDuration sur la voie. Voici la contrainte qui modélise cette amélioration: 

\begin{equation*}
    \begin{split}
      & \bigwedge_{(g_1, g_2), i > 0, t} 
        \bigwedge_{i+1\leq i_2<i+TravelDuration} 
        \big( \neg V_{g_1, g_2, t, i-1} \land  V_{g_1, g_2, t, i}\big)
        \implies   
        \neg \big( 
        \bigwedge_{i + 1 < i' \leq i + 2 * TravelDuration} 
        V_{g_1, g_2, t, i'} \big) \\
    \end{split}
    \end{equation*}



\section{Bonus}


\end{document}