\documentclass{article}
\usepackage[utf8]{inputenc}
\usepackage{amsmath}


\begin{document}
 \begin{equation*}
\bigvee_{a \in \{1, 2, 3\}}
\bigvee_{b \in \{7, 8\}}
c_{a, b}
\end{equation*}

\begin{equation*}
c_{1, 7} \lor
c_{1, 8} \lor
c_{2, 7} \lor
c_{2, 8} \lor
c_{3, 7} \lor
c_{3, 8} 
\end{equation*}





\section{Contraintes Explicites}

\begin{itemize}

\item Contrainte 1
    \begin{equation*}
    \begin{split}
      & \big(\bigwedge_{(g,g'), g \neq g'} 
      \bigwedge_{TimeWindow [i_s,i_e] non-glissante}
      \bigvee_{i, i' = i+TravelDuration \in [i_s,i_e] }
      \bigvee_{t}
      \big[\neg G_{t,i,g} \lor G_{t,i',g'} \big] \big)\\
      \land \\
      &\big(\bigwedge_{TimeWindow:[i_s,i_e]}
      \big[\bigvee_{g, t, i \in TimeWindow} G_{t, i, g} \big]\big) \\
    \end{split}
    \end{equation*}
    
\item Contrainte 1 - v2
    \begin{equation*}
    \begin{split}
      & 
      \bigwedge_{(g,g'), g \neq g'} 
      \bigwedge_{TimeWindow [i_s,i_e]}
      \bigvee_{i \in [i_s,i_e]}
      \bigvee_{i < i' \leq min(TimeSlots - 1, i + TravelDuration)}
      \bigvee_{t}
      \big[T_{t,i,g,i', g'}] \\
    \end{split}
    \end{equation*}

Les Time Windows utilisées sont non-glissantes (elles ne se superposent pas).
    
\item Contrainte 2
    \begin{equation*}
    \begin{split}
      & \bigwedge_{t_1 \neq t_2}
      \bigwedge_{i}
      \bigwedge_{(g,g'), g \neq g'} \big[
      \neg V_{t_1,i,(g,g')} \lor \neg V_{t_2,i,(g,g')} \big] \\
    \end{split}
    \end{equation*}
    
\item Contrainte 3
    \begin{equation*}
    \begin{split}
      & \bigwedge_{t}
      \bigwedge_{i}
      \bigwedge_{(g_1,g_2), (g_2, g_3)}  \big[
      \neg V_{t,i,(g_1,g_2)} \lor \neg V_{t,i+1,(g_2,g_3)} \lor (Rapide(t) \land Petite(g_2) \big] \\
    \end{split}
    \end{equation*}
    
\item Contrainte 4
    \begin{equation*}
    \begin{split}
      & \bigwedge_{t, g, i} \big[
      \big(G_{g, t, i-1} \lor \neg G_{g, t, i} \big)
      \lor 
      \big(\bigwedge_{i+1 \leq i_2<i+TWait} G_{g, t, i_2} \big) \big] \\
    \end{split}
    \end{equation*}
    
\item Contrainte 4 (équivalent en FNC)
    \begin{equation*}
    \begin{split}
      & \bigwedge_{t, g, i} \bigwedge_{i+1 \leq i_2<i+TWait} \big[
      \big(G_{g, t, i-1} \lor \neg G_{g, t, i} \lor G_{g, t, i_2} \big) \\
    \end{split}
    \end{equation*}

\item Contrainte 5
    \begin{equation*}
    \begin{split}
      & \bigwedge_{(g_1, g_2), t, i} \big[ V((g_1, g_2), t, i-1) \lor \neg V((g_1, g_2), t, i) \big]  \\
      & \lor\\
      & \big[ \bigwedge_{i+1<=i_2<i+TravelDuration} V((g_1, g_2), t, i_2) \big]
    \end{split}
    \end{equation*}
    
  
\item Contrainte 5 (équivalent en FNC)
    \begin{equation*}
    \begin{split}
      & \bigwedge_{(g_1, g_2), t, i} \bigwedge_{i+1<=i_2<i+TravelDuration} \big[ V((g_1, g_2), t, i-1) \lor \neg V((g_1, g_2), t, i) \lor V((g_1, g_2), t, i_2)\big] \\
    \end{split}
    \end{equation*}
    
\item Contrainte 6

    \begin{equation*}
    \begin{split}
        & \bigwedge_{g, i} \bigwedge_{Ensemble De Trains E \{t_1, t_2, \cdot, t_{cg}, t_{c_g+1}\}} \bigvee_{t \in E} \neg G_{t, g, i}
    \end{split}
    \end{equation*}




\section{Contraintes Implicites}

\item Contrainte 1 : Tout train doit être quelque part

\begin{equation*}
    \begin{split}
      & \bigwedge_{t, i}
      \big[\bigvee_{g} \big(G_{g, i, t} \big) \lor \bigvee_{(g_1, g_2)} \big(V_{(g_1, g_2) i, t} \big) \big] \\
    \end{split}
    \end{equation*}


\item Contrainte 2 : Un train ne peut pas être dans 2 gares différentes au même moment

\begin{equation*}
    \begin{split}
      & \bigwedge_{g_1, g_2, i>0, t, g_1 \neq g_2 }
      \big[\neg G_{g_1, i, t} \lor \neg G_{g_2, i, t} \big] \\
    \end{split}
    \end{equation*}
    
\item Contrainte 3 :Un train ne peut pas être sur 2 segment différents au même moment

\begin{equation*}
    \begin{split}
      & \bigwedge_{(g_1, g_2), (g_3, g_4), i>0, t, g_1 \neq g_3, g_2 \neq g_4 }
      \big[\neg V_{(g_1, g_2), i, t} \lor \neg V_{(g_3, g_4), i, t} \big] \\
    \end{split}
    \end{equation*}

\item Contrainte 4 :Un train ne peut pas être sur un segment et une gare au même moment

\begin{equation*}
    \begin{split}
      & \bigwedge_{(g_1, (g_2, g_3), i>0, t}
      \big[\neg G_{g_1, i, t} \lor \neg V_{(g_2, g_3), i, t} \big] \\
    \end{split}
    \end{equation*}


\item Contrainte 5 : Lorsqu'on sort d'un segment A-B, soit on arrive à gare B soit on est sur un segment B-C

\begin{equation*}
    \begin{split}
      & \bigwedge_{(g_1, g_2), t, i}
      \big[
      \big(\neg V_{(g_1, g_2), i-1, t} \lor V_{(g_1, g_2), i, t} \big)
      \lor
      \big(G_{g_2, t, i} \lor \bigvee_{g_3,\ g_3\ \text{liée à}\ g_2} \big(V_{(g_2, g_3), i, t}\big) \big)
      \big]
      \end{split}
    \end{equation*}


\item Contrainte 6 :Lorsqu'un train sort d'une gare A, il doit être sur un segment qui part de A

\begin{equation*}
    \begin{split}
      & \bigwedge_{g_1, t, i}
      \big[
      \big(\neg G_{g_1, i-1, t} \lor G_{(g_1, i, t} \big)
      \lor
      \big(\bigvee_{g_2,\ g_1\ \text{liée à}\ g_2} V_{(g_1, g_2), i, t}\big) 
      \big]\\
    \end{split}
    \end{equation*}
    
\item Contrainte 7 : contraintes pour cohérence de la variable T (partie 1)
\begin{equation*}
    \begin{split}
      & \bigwedge_{t, g, g', i, i', g \neq g', i < i'}
      (\neg T_{t,i,g,i', g'} \lor G_{g, i, t} ) \land (\neg T_{t,i,g,i', g'} \lor  G_{g', i', t} )\\
    \end{split}
    \end{equation*}

\item Contrainte 8 : contraintes pour cohérence de la variable T (partie 2)
\begin{equation*}
    \begin{split}
      & \bigwedge_{t, g, g', i, i', g \neq g', i < i'}
      \neg G_{g, i, t} \lor \neg G_{(g', i', t} \lor T_{t,i,g,i', g'} \\
    \end{split}
    \end{equation*}


\end{itemize}
\end{document}